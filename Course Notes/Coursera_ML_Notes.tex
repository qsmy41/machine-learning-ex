\documentclass[11pt]{report}
\usepackage[thinc]{esdiff} % for typesettign derivatives
\usepackage{amsthm} % provides an enhanced version of LaTex's \newtheorem command
\usepackage{mdframed} % framed environments that can split at page boundaries
\usepackage{enumitem} % bulletin points or other means of listing things
\usepackage{amssymb} % for AMS symbols
\usepackage{amsmath} % so as to use align
\usepackage{latexsym} % so as to use symbols like \leadsto
\usepackage{mathrsfs} % for using mathscr for char like operators
\usepackage{authblk} % for writing affiliations
\usepackage{graphicx} % for importing images
\graphicspath{{./images/}} % for the path to images, also always put label behind captions
\usepackage{textcomp} % for using degree symbol
\usepackage{hyperref} % for clickable link in the pdf

\theoremstyle{definition}
\mdfdefinestyle{defEnv}{%
  hidealllines=false,
  nobreak=true,
  innertopmargin=-1ex,
}

\pagestyle{headings}
\author{Lectured by Andrew Ng}
\title{Coursera Machine Learning Notes}
\affil{Typed by Aris Zhu Yi Qing}
\begin{document}
\maketitle
\tableofcontents

\chapter{Neural Networks}

\section{Notations}
\begin{itemize}
    \item $a^{(i)}_{k}$: activation energy at layer $i$ of element $k$.
    \item $a^{(L)}$: the activation energy of the last layer,
        i.e.\ the values in the output layer.
    \item $\Theta_{ij}^{(l)}$: the weight 
        between the element $i$ in layer $l + 1$ 
        and the element $j$ in layer $l$.
    \item $y$ (or $Y$): target output(s).
    \item $\hat{y}$ (or $\hat{Y}$): derived output(s), equivalent to $a^{(L)}$.
\end{itemize} 

\section{Backward Propagation Derivation/Proof}
As seen in the previous chapters, after deriving $J(\Theta)$, i.e.\ 
\[
    J(\Theta) = \frac{1}{m} \sum_{i=1}^{m} \sum_{k=1}^{K} 
    [-y^{(i)}_{k}\log{((h_{\theta}(x^{(i)}))_{k})
        - (1 - y_{k}^{(i)})\log{(1 - (h_{\theta}(x^{(i)}))_{k})],
\]or\[
J(\Theta) = \frac{1}{m} *
\mathscr{S}(- Y \;.* \log{(\hat{Y})} - (1 - Y) \;.* \log{(1 - \hat{Y})}),
\]where $\mathscr{S}$ refers to the sum of all the elements in the matrix,
and $.*$ refers to element-wise multiplication.
The next challenge is, as with previous chapters, to find out
what represents \[
    \frac{\partial J}{\partial \Theta_{ij}^{(l)}}.
\]
For the sake of illustration,
we are ignoring the regularization terms and dimension incompatibility.
\medskip
\\Now the following is where the magic begins. 
It turns out that
\begin{align*}
    \Delta^{(l)} := \frac{\partial J}{\partial \Theta^{(l)}}
    &= \frac{\partial z^{(l + 1)}}{\partial \Theta^{(l)}} 
    \frac{\partial J}{\partial z^{(l + 1)}} \\
    &= \frac{\partial (\Theta^{(l)} * a^{(l)})}{\partial \Theta^{(l)}}
    \delta^{(l + 1)} \\
    &= a^{(l)} \delta^{(l + 1)},
\end{align*} 
where
\begin{align*}
    \delta^{(l)} := \frac{\partial J}{\partial z^{(l)}}
    &= \frac{\partial z^{(l + 1)}}{\partial z^{(l)}}
    \frac{\partial J}{\partial z^{(l + 1)}} \\
    &= \frac{\partial (\Theta^{(l + 1)} * g(z^{(l)}))}{\partial z^{(l)}} 
    \delta^{(l + 1)} \\
    &= {\Theta^{(l + 1)}}^{(T)} \delta^{(l + 1)} g'(z^{(l)})
\end{align*} 
and
\begin{align*}
    \delta^{(L)}
    &= \frac{\partial J}{\partial z^{(L)}} \\
    &= \frac{\partial \hat{y}}{\partial z^{(L)}} 
    \frac{\partial J}{\partial \hat{y}},
\end{align*} 
so we need to figure out the value of the two fractions
to obtain the ``terminating value'' of the $\delta$s.
On the other hand, we can also derive this by figuring out
\begin{equation}\label{eqn:1}
    \delta^{(L)}_k = \frac{\partial \hat{y}_k}{\partial z^{(L)}_k}
    \frac{\partial J}{\partial \hat{y}_k}
\end{equation} 
and line them up to return to vector form, $\delta^{(L)}$.
For the sake of clarity of proof, we will use equation~\eqref{eqn:1}.
\bigskip
\\ Let's tackle the first fraction.
Since $\hat{y}_k = \sigma(z^{(L)}_k)$, and
\begin{align*}
    \sigma'(z)
    &= (-1)(1 + e^{-z})^{-2}\frac{\partial (1 + e^{-z})}{\partial z}\\
    &= \frac{1}{1 + e^{-z}} \cdot \frac{1 + e^{-z} - 1}{1 + e^{-z}} \\
    &= \sigma(z)(1 - \sigma(z)),
\end{align*} 
we can derive that
\begin{equation}\label{eqn:2}
    \frac{\partial \hat{y}_k}{\partial z^{(L)}_k} = \hat{y}_k(1 - \hat{y}_k).
\end{equation} 
\medskip
\\Now let's tackle the second fraction. 
From the provided $J(\Theta)$,
we look at the $k$-th output,
and calculate the derivative of the loss with respect to its activation energy:
(In the following equation, $y_k$ is abbreviated to $y$, 
$\hat{y}_k$ is abbreviated to $\hat{y}$ for clarity)
\begin{align*}
    \frac{\partial J}{\partial \hat{y}}
    &= -\frac{y}{\hat{y}} + \frac{1 - y}{1 - \hat{y}} \\
    &= \frac{\hat{y}(1 - y) - (1 - \hat{y})y}{\hat{y}(1 - \hat{y})} \\
    &= \frac{\hat{y} - y}{\hat{y}(1 - \hat{y})}.
\end{align*} 
i.e.
\begin{equation}\label{eqn:3}
    \frac{\partial J}{\partial \hat{y}_k} 
    = \frac{\hat{y}_k - y_k}{\hat{y}_k(1 - \hat{y}_k)}
\end{equation} 
Substituting equations~\eqref{eqn:2} and \eqref{eqn:3} into~\eqref{eqn:1},
and convert into vectorized form, we can get that\[
    \delta^{(L)} = \hat{y} - y.
\]


\end{document}
